% Template for a Computer Science Tripos Part II project dissertation
\documentclass[12pt,a4paper,twoside,openright]{report}
\usepackage[pdfborder={0 0 0}]{hyperref}    % turns references into hyperlinks
\usepackage[margin=25mm]{geometry}  % adjusts page layout
\usepackage{graphicx}  % allows inclusion of PDF, PNG and JPG images
\usepackage{verbatim}
\usepackage{docmute}   % only needed to allow inclusion of proposal.tex

\raggedbottom                           % try to avoid widows and orphans
\sloppy
\clubpenalty1000%
\widowpenalty1000%

\renewcommand{\baselinestretch}{1.1}    % adjust line spacing to make
                                        % more readable

\begin{document}

\bibliographystyle{plain}


%%%%%%%%%%%%%%%%%%%%%%%%%%%%%%%%%%%%%%%%%%%%%%%%%%%%%%%%%%%%%%%%%%%%%%%%
% Title


\pagestyle{empty}

\rightline{\LARGE \textbf{Raahil Shah}}

\vspace*{60mm}
\begin{center}
\Huge
\textbf{MSMPTCP using Random Linear Network Coding} \\[5mm]
Computer Science Tripos -- Part II \\[5mm]
Churchill College \\[5mm]
\today  % today's date
\end{center}

%%%%%%%%%%%%%%%%%%%%%%%%%%%%%%%%%%%%%%%%%%%%%%%%%%%%%%%%%%%%%%%%%%%%%%%%%%%%%%
% Proforma, table of contents and list of figures

\pagestyle{plain}

\chapter*{Proforma}

{\large
\begin{tabular}{ll}
Name:               & \bf Raahil Shah                       \\
College:            & \bf Churchill College                     \\
Project Title:      & \bf MSMPTCP using Random Linear Network Coding \\
Examination:        & \bf Computer Science Tripos -- Part II, May 2016  \\
Word Count:         & \bf 0\footnotemark[1]
                      (well less than the 12000 limit)  \\
Project Originator: & Dr J. Crowcroft                    \\
Supervisor:         & Dr J. Crowcroft                    \\ 
\end{tabular}
}
\footnotetext[1]{This word count was computed
by \texttt{detex diss.tex | tr -cd '0-9A-Za-z $\tt\backslash$n' | wc -w}
}
\stepcounter{footnote}


\section*{Original Aims of the Project}

The original aim of this project was to combine Multipath TCP (MPTCP) with the Network Coding paradigm while allowing for many-to-one flows, i.e. multiple sources. While Multipath TCP provides greater resource usage and often near optimal load balancing by spreading information across multiple flows, the addition of network coding, by spreading information over both packets and flows, aims to maximise the degrees of freedom available from resource pooling. The protocol was to be testing on simulated wireless topologies in a network simulator to evaluate its performance in terms of throughput and robustness against failures. 

\section*{Work Completed}

In this project I provide a simple design and implementation of network coding using the random linear network coding technique. The coding layer is run in the chosen network simulator: ns-3 using the Kodo library for C++ and ns3. Due to challenges (mentioned below) the network coding is run over UDP, rather than MPTCP, in the application layer. The simulations achieve reliable file transfer over UDP and increased reliable throughput over lossy wireless channels.

%TODO include eval

\section*{Special Difficulties}

After researching MPTCP and MPTCP with network coding (MPTCP/NC) protocols, implementing the multisource and network coding extensions seemed too ambitious, especially in the time frame of this project. Further, the main challenge with using MPTCP for this project was the lack of a publicly available MPTCP module implemented for ns3. In consultation with my supervisor and the overseers I decided to follow the backup strategy laid out in the proposal [Appendix \ref{ch:proposal}] which was to use the coding layer over UDP instead of MPTCP. 
 
\newpage
\section*{Declaration}

I, Raahil Shah of Churchill College, being a candidate for Part II of the Computer
Science Tripos, hereby declare
that this dissertation and the work described in it are my own work,
unaided except as may be specified below, and that the dissertation
does not contain material that has already been used to any substantial
extent for a comparable purpose.

\bigskip
\leftline{Signed }

\medskip
\leftline{Date }

\tableofcontents

\listoffigures

\newpage
\section*{Acknowledgements}


%%%%%%%%%%%%%%%%%%%%%%%%%%%%%%%%%%%%%%%%%%%%%%%%%%%%%%%%%%%%%%%%%%%%%%%
% now for the chapters

\pagestyle{headings}

\chapter{Introduction} \label{ch:intro}

\section{Motivation for the Project}

The widespread success and rapid growth of wireless technologies in communication networks make it clear that wireless networking is closely linked to the present and future of communications. However, the Transmission Control Protocol (TCP) presently used in a wide variety of applications for reliable data transfer, suffers greatly in performance from the challenges presented by today's wireless networks. These include the multipath nature of networks, for example with devices having multiple wireless interfaces, datacenters having redundant paths and large server farms employing multihoming, as well as the lossy nature of wireless links resulting in the need for complex management schemes in the protocol to ensure reliable transmission. Recent research work shows that these challenges can be addressed by the the concept of network coding, originally introduced by Ahlswede et al \cite{ahlswede}, and the new Multipath TCP (MPTCP)\cite{mptcp-ietf}, an ongoing effort of the Internet Engineering Task Force (IETF) MPTCP working group.

Network Coding is a field of information theory that presents an alternative to the conventional store-and-forward routing paradigm and allows us to extend existing resource pooling from information streams sharing network resources to spreading the information over the data packets as well. Using network coding, nodes can modify the data of the packets across the network, either at the source or at an intermediate node (router) which can lead to a number of performance benefits, including reducing the number of transmissions needed for a reliable transfer providing increased throughput, increasing robustness to packet losses and link failures as well as potential implicit security mechanisms from the transmission of coded rather than raw packet data. Summarised developments on network coding and its benefits can be found in \cite{nc-intro}.

MPTCP is a major modification of TCP that allows a single transport layer connection to use multiple paths simultaneously. It's application can be seen in many real world scenarios, for example a cell phone or laptop with multiple network interfaces (WiFi/3G) downloading a file. MPTCP allows the device to set up multiple `subflows' for a single MPTCP session. This not only leads to better load balancing across the flows which significantly benefits throughput especially for lossy wireless networks like WiFi and 3G, but also results in smoother handover in the case that one of the interfaces goes down as compared to an application layer handover over TCP. \cite{mptcp-over} presents a succinct overview of the implementation and performance of the protocol.  

Several suggestions on combining network coding with MPTCP or TCP have been proposed in literature (\cite{mptcp-nc-mesh}, \cite{mptcpnc}, \cite{tcpnc}). They indicate that MPTCP/NC provides better throughput, robustness to packet losses and scheduling of packets on each subflow, under certain experimental conditions and assumptions. These results are enough to show us that integrating network coding in today's protocol stack has great potential to strengthen and enhance wireless communications. 

\subsection{Project Description}

%TODO: Include eval

This project successfully demonstrates the benefits of network coding with respect to reliable transfer, throughput gain and robustness to packet losses on wireless networks. This is achieved by implementing a simple network coding protocol which is described in Section \ref{sec:nclayer}. The protocol is demonstrated to run in the application layer of the OSI protocol stack over a standard, real world network stack comprising of User Datagram Protocol (UDP) in the transport layer, Internet Protocol version 4 (IPv4) in the network layer and ad hoc 802.11 WiFi in the link and physical layers. 

The network coding scheme implemented is Intra-flow network coding using Random Linear Coding (RLC), discussed in Section \ref{sec:nctheory}, which mixes or codes datagrams from the same UDP flow. The coding layer is used to implement two techniques of Intra-flow network coding:
\begin{description}
 	\item[Random Linear Source Coding] where only the source node(s) code the original (native) packets to transmit a sequence of coded packets into the network while the intermediate nodes simply relay the coded packets received onwards using the conventional store-and-forward paradigm.
 	\item[Random Linear Network Coding] where along with RLC at the source node(s), the intermediate relay nodes may perform `re-coding' operations on the coded packets received, forwarding new coding packets to the next hop in the network.
 \end{description}
 The performance of each technique is evaluated in Sections \ref{sec:rlsc}, \ref{sec:rlnc}. We also look at the effects of varying the RLC parameters of the size of the Galois Field used to select coding coefficients and the 


\chapter{Preparation} \label{ch:prep}

\section{Network Coding} \label{sec:nctheory}
\section{Network Protocols: MPTCP, TCP/NC, UDP}
\section{Network Simulator: ns3}

\chapter{Implementation} \label{ch:imp}

\section{Network Coding Layer} \label{sec:nclayer}

\subsection{Protocol Description}
\subsection{Encoding}
\subsection{Decoding}

\section{Reliable Transfer over UDP}

\section{Kodo Network Coding Libary}

\section{Simulation in ns-3} \label{sec:ns3}

\chapter{Evaluation} \label{ch:eval}

\section{Reliable File Transfer} \label{sec:relUDP}

\section{Varying Field and Generation Size} \label{sec:fieldgen}

\section{Random Linear Source Coding} \label{sec:rlsc}

\section{Random Linear Network Coding} \label{sec:rlnc}

\chapter{Conclusion} \label{ch:conc}

%%%%%%%%%%%%%%%%%%%%%%%%%%%%%%%%%%%%%%%%%%%%%%%%%%%%%%%%%%%%%%%%%%%%%
% the bibliography
\addcontentsline{toc}{chapter}{Bibliography}
\bibliography{refs}

%%%%%%%%%%%%%%%%%%%%%%%%%%%%%%%%%%%%%%%%%%%%%%%%%%%%%%%%%%%%%%%%%%%%%
% the appendices
\appendix

\chapter{Project Proposal}
\label{ch:proposal}

% Note: this file can be compiled on its own, but is also included by
% diss.tex (using the docmute.sty package to ignore the preamble)
\documentclass[12pt,a4paper,twoside]{article}
\usepackage[pdfborder={0 0 0}]{hyperref}
\usepackage[margin=25mm]{geometry}
\usepackage{graphicx}

\renewcommand\labelitemi{--}

\begin{document}

\vfil

\centerline{\Large Computer Science Project Proposal}
\vspace{0.4in}
\centerline{\Large MSMPTCP: Multisource Multipath TCP using Random Linear Codes}
\vspace{0.4in}
\centerline{\large R. Shah, Churchill College}
\vspace{0.3in}
\centerline{\large Originator: Dr J. Crowcroft}
\vspace{0.3in}
\centerline{\large \today}

\vfil

\noindent
{\bf Project Supervisor:} Dr J. Crowcroft
\vspace{0.2in}

\noindent
{\bf Director of Studies:} Dr J. K. Fawcett
\vspace{0.2in}
\noindent
 
\noindent
{\bf Project Overseers:} Dr T. G. Griffin  \& Dr P. Lio'


% Main document

\section*{Introduction, The Problem To Be Addressed}
MPTCP is an extension to TCP that sprung from the observation that modern networks are quite often multipath, as seen in server farms that use multihoming, datacenters with several redundant paths between servers and in modern host devices such as mobile devices that have multiple network interfaces (for example, WiFi and 3G). TCP by itself cannot take advantage of this to load balance across multiple paths. MPTCP, on the other hand, allows for multiple subflows to be set up for a single session, reaping the benefits of resource pooling using multiple paths in the network, all transparent to the application. \cite{mptcp}

This project aims to extend MPTCP, to incorporate network coding and to allow for multiple sources for the connection. Adding multisourcing to MPTCP has its advantages in cases where there is a lot of replication of data, for example within and across many datacenters. The addition of network coding to MPTCP spreads information over both packets and flows, maximising the degrees of freedom that resource pooling can enjoy. Further network coding can increase throughput and improve robustness against failures, especially in lossy networks (for example, wireless networks). \cite{nc-tcp}

\section*{Starting Point}

This project will build on an existing userspace implementation of MPTCP to extend it to MSMPTCP with random linear coding.

\section*{Resources Required}

\begin{description}
  \item[Hardware] Personal computer: MacBook Pro (Mid 2014), 8GB 1600 MHz DDR3 RAM, 2.8 GHz Intel Core i5 Processor, running Mac OS X 10.11.
  \item[Software] Network Simulator (ns-2 or ns-3).
  \item[Backup] All the project files, i.e. source code, research and authored documents, will be stored under a project directory on my personal computer which is synced to cloud storage (Google Drive). For an offline solution, the project directory will be backed up (with at least weekly frequency) to an external hard disk using Time Machine. I will also maintain (at least weekly updated) a backup of the project directory on the university MCS to be used in case my personal machine cannot be used for project work. I would require an extension of storage space on the MCS for this purpose. All source code and written documents will be under git version control and pushed to a private repository on GitHub. Commits and pushes will be made at least daily and each time a significant change to any code or document is made. 
\end{description}

\section*{Work to be done}

The project breaks down into the following sub-projects:

\begin{enumerate}
  \item Implementing a network coding layer using random linear codes. This will involve implementing a sender side coding module and a receiver side decoding module. The coding implementation will be tested on the standard butterfly network topology in the network simulator. 
  \item Extending MPTCP to MSMPTCP with network coding. This sub-project will involve extending the MPTCP implementation to include the network coding layer implemented in sub-project 1 along with adding support for multiple sources. The resulting MSMPTCP implementation will be tested in the network simulator on a butterfly topology with multiple sources. In case extending MPTCP proves unfeasible within the time frame of this project, the fallback task for this sub-project would be to add the multisource and network coding extensions to a simpler base protocol such as UDP. The fallback protocol will be tested for the same cases as MSMPTCP and the same success criterion for the sub-project will apply.
  \item Evaluating the performance of MSMPTCP with network coding against MPTCP and TCP. The evaluation will take the form of graphs of throughput obtained for MSMPTCP with random linear codes against that obtained for MPTCP and TCP, for tests run on the network simulator. Performance of MPTCP and MSMPTCP could also be compared for varying number of subflows. 
\end{enumerate}

\section*{Success Criterion for the Main Result}

The success criterion for sub-project 1 would be a set of successful tests of the implemented network coding layer in the network simulator on the standard butterfly topology. Similarly, sub-project 2 can be considered successful if the MSMPTCP with network coding implementation succeeds for a set of network simulator tests on the standard butterfly topology along with a butterfly topology with multiple sources. 

\section*{Possible Extensions}

\begin{enumerate}
  \item Re-encoding packets at intermediate nodes which is useful in combating against packet losses.
  \item Testing MSMPTCP with random linear coding with various different network topologies in network simulator.
\end{enumerate}

\section*{Timetable: Workplan and Milestones to be achieved.}

Planned starting date is 22/10/2015.

\begin{enumerate}

\item {\bf Michaelmas weeks 3, 4} (22/10/2015 - 04/11/2015)

Research on network coding theory with specific focus on random linear codes and implementations of network coding with TCP.
\begin{itemize}
  \item {\em Deliverable:} Summary of theory learned and an implementation strategy for random linear coding in a LaTeX document for inclusion in the progress report. 
  \item {\em Milestone:} Submit the deliverable document to supervisor for review and feedback.
\end{itemize}

\item {\bf Michaelmas weeks 5, 6} (5/10/2015 - 18/11/2015)
Install ns-2/ns-3 and learn to use it to generate network topologies and simulate protocols.

\begin{itemize}
  \item {\em Deliverable:} Simulation code and results for demo topologies and protocols.
  \item {\em Milestone:} Demonstrate topology generation and simulations to supervisor.
\end{itemize}

\item {\bf Michaelmas weeks 7, 8} (19/11/2015 - 02/12/2015)

Work on random linear coding implementation for sub-project 1.
\begin{itemize}
  \item {\em Deliverable:} Source code and documentation.
  \item {\em Milestone:} Unit test all code modules written.
\end{itemize}

\item {\bf Michaelmas vacation weeks 1, 2} (03/12/2015 - 16/12/2015)

Work on random linear coding implementation for sub-project 1.
\begin{itemize}
  \item {\em Deliverable:} Source code and documentation.
  \item {\em Milestone:} Unit test all code modules written. Have a feature complete network coding layer implementation.
\end{itemize}

\item {\bf Michaelmas vacation weeks 3, 4} (17/12/2015 - 30/12/2015)

Test the network coding layer on the standard butterfly topology in network simulator.
\begin{itemize}
  \item {\em Deliverable:} Simulation code and results.
  \item {\em Milestone:} Have a tested, feature complete network coding layer implementation.
\end{itemize}

\item {\bf Michaelmas vacation weeks 5, 6} (31/12/2015 - 13/01/2015)

Slack time to complete any outstanding work for sub-project 1 and/or work on possible extensions.
Write the progress report. 
\begin{itemize}
  \item {\em Deliverable:} Draft progress report.
  \item {\em Milestone:} Have completed sub-project 1. Submit the draft progress report to supervisor and DoS for review and feedback.
\end{itemize}

\item {\bf Lent weeks 1, 2} (14/01/2016 - 27/01/2016)

Research on MPTCP and get familiar with the MPTCP implementation code.
\begin{itemize}
  \item {\em Deliverable:} Summary of research done and outline to extend MPTCP in a document.
  \item {\em Milestone:} Submit the deliverable document to supervisor for review and feedback. Submit the final progress report.
\end{itemize}

\item {\bf Lent weeks 3, 4} (28/01/2016 - 10/02/2016)

Work on MSMPTCP implementation for sub-project 2 using the network coding layer implementation from sub-project 1.
\begin{itemize}
  \item {\em Deliverable:} Source code and documentation.
  \item {\em Milestone:} Successful unit tests for any code modules written.
\end{itemize}

\item {\bf Lent weeks 5, 6} (11/02/2016 - 24/02/2016)

Work on MSMPTCP implementation for sub-project 2 using the network coding layer implementation from sub-project 1.
\begin{itemize}
  \item {\em Deliverable:} Source code and documentation.
  \item {\em Milestone:} Successful unit tests for any code modules written. Have a feature complete MSMPTCP implementation.
\end{itemize}

\item {\bf Lent weeks 7, 8} (25/02/2016 - 09/03/2016)

Test the MSMPTCP protocol with standard and multiple source butterfly topology in network simulator.
\begin{itemize}
  \item {\em Deliverable:} Simulation code and results.
  \item {\em Milestone:} Have a tested, feature complete MSMPTCP with random linear coding implementation.
\end{itemize}

\item {\bf Easter vacation weeks 1, 2} (10/03/2016 - 23/03/2016)

Slack time to complete any outstanding work for sub-project 2 and/or work on possible extensions.
\begin{itemize}
  \item {\em Milestone:} Have completed sub-projects 1 and 2.
\end{itemize}

\item {\bf Easter vacation weeks 3, 4} (24/03/2015 - 06/04/2015)

Work on drafting the main chapters of the dissertation.
\begin{itemize}
  \item {\em Deliverable:} Draft dissertation chapters.
  \item {\em Milestone:} Submit the draft to supervisor and DoS for review and feedback.
\end{itemize}

\item {\bf Easter vacation weeks 5, 6} (07/04/2015 - 20/04/2015)

Work on revising and completing draft dissertation incorporating feedback from supervisor and DoS. Perform evaluation (sub-project 3) to be included in the dissertation.
\begin{itemize}
  \item {\em Deliverable:} Draft dissertation chapters including evaluation.
  \item {\em Milestone:} Submit the draft to supervisor and DoS for review and feedback.
\end{itemize}

\item {\bf Easter weeks 1, 2} (21/04/2016 - 04/05/2016)

Finish writing the dissertation incorporating any final suggestions. 
\begin{itemize}
  \item {\em Deliverable:} Final version of dissertation. 
  \item {\em Milestone:} Submit the final version of the dissertation to supervisor and DoS for review and feedback. 
\end{itemize}

\item {\bf Easter week 3} (05/05/2016 - 11/05/2016)

Proof read the dissertation and submit it. 
\begin{itemize}
  \item {\em Deliverable:} Two paper copies and a PDF file of the dissertation.
  \item {\em Milestone:} Submit the dissertation copies and code files to Student Administration.
\end{itemize}

\end{enumerate}

% References

\begin{thebibliography}{9}

\bibitem{mptcp}
  O. Bonaventure, M. Handley, C. Raiciu,
  ``An overview of Multipath TCP'',
  \emph{The Magazine of the USENIX Association},
  2012.

\bibitem{nc-tcp}
  J. K. Sundararajan, D. Shah, M. Médard, S. Jakubczak, M. Mitzenmacher, J. Barros,
  ``Network Coding Meets TCP: Theory and Implementation'',
  \emph{Invited Paper, Proceedings of the IEEE},
  March 2011,
  pp. 490-512.

\end{thebibliography}


\end{document}


\end{document}
